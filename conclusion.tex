\chapter{Conclusion}
\label{ch:conclusion}

The conclusion of your thesis.

Note that you can also add Appendices. 
Appendices are introduced with the command \verb|\appendix| (see at the end of the \verb|thesis.tex| file) and each appendix is then entered as a chapter that can then be cross-referenced, like the example \Cref{app:AppendixA} in this thesis template.

Finally, notice that this chapter content is in a separate \verb|.tex| file called \verb|conclusion.tex|, and is included using the command command \verb|\include|.
Splitting the thesis content is separate \verb|.tex| files | typically one per chapter | simplifies the main document. 
And more beneficially, once the content of a chapter is completed, you can prevent it from being systematically compiled by commenting it out in the main \verb|thesis.tex| file. 
By only uncommenting the \verb|\include| command of the chapter you are currently working on, you ensure Overleaf compiles only that chapter, which speeds up compilation considerably. 
Only at the very end, do you then uncomment all \verb|\include| commands of all chapters in order to compile the complete thesis document.